\chapter{Conclusions} \label{chap:conclusions}
\section{Summary} \label{sec:conclusions_summary}
Several analyses have been presented of events with a monophoton signature selected out of 35.9\fbinv\ of 13\unit{TeV} \Pp\Pp\ collision
data collected by the CMS experiment at CERN in 2016. The primary expected SM contribution to such events is
the \zinvg\ process, and a measurement of the production cross section of this process at 13\unit{TeV} has been obtained at state-of-the-art
precision. In particular, the cross section measurement for events with $\pTgamma > 600\unit{GeV}$, found here to be
$1.18 \pm 0.04\mathrm{(syst)} \pm 0.36\mathrm{(stat)}$, has roughly half the percentage uncertainty of the next-most-precise
measurement in this regime, published by ATLAS in 2018~\cite{ref:CERN-EP-2018-220}. This is the first measurement of the \zinvg\ cross section
at 13\unit{TeV} reported by CMS, and the first CMS measurement of this cross section unfolded into separate bins of \pTgamma.

The results presented here do not show any $5\sigma$ deviations from SM predictions in favor of any other model.
The 95\% CL lower limit on the suppression scale $\Lambda$ of the DM-EWK EFT is 850\unit{GeV} for small \mdm\ values, the strongest
limits on this model to date. This represents an increase of 60\unit{GeV} over the strongest ATLAS limit on $\Lambda$~\cite{ref:epjc/s10052-017-4965-8},
and an increase of 250\unit{GeV} over the prior CMS results based on 12.9\fbinv\ of 2016 data~\cite{ref:JHEP10(2017)073}.
For DM simplified models, mediator masses up to 950\unit{GeV} are excluded at 95\% CL for small \mdm\ values,
also an increase of 250\unit{GeV} over the previous CMS analysis. The strongest observed monophoton limit on \mmed\ remains
1200\unit{GeV}, that of the ATLAS experiment based on their analysis of 2016 data~\cite{ref:epjc/s10052-017-4965-8}.
The 95\% CL lower limit on the ADD mass scale \mD\ is 2.85\unit{TeV} for $n = 3$ extra dimensions, rising to 2.90\unit{TeV} for $n = 6$.
This represents an increase of 410\unit{GeV} at high $n$ over the prior CMS results based on 12.9\fbinv\ of 2016 data~\cite{ref:JHEP10(2017)073},
and up to 560\unit{GeV} at low $n$. These are the strongest ADD limits in the monophoton channel to date.
The DM and ADD results presented this thesis have been published in the Journal of High Energy Physics as ``Search for new physics in final states with
a single photon and missing transverse momentum in proton-proton collisions at $\sqrt{s} = 13\unit{TeV}$''~\cite{ref:JHEP02(2019)074}.

\section{Outlook} \label{sec:conclusions_outlook}
Roughly four times as much data as analyzed here has been collected by CMS in the years 2016, 2017, and 2018 combined,
and the analysis of this combined data set is currently in progress. Additional \Pp\Pp\ collision data at 13\unit{TeV} will significantly
improve the precision of the \zinvg\ cross section measurement, especially for high \pTgamma:
a factor of four increase in observed data is expected to scale the percentage statistical uncertainty by \textasciitilde$1/\sqrt{4} = 1/2$.
Several times more data would be required for the statistical uncertainty in the highest \pTgamma\ bin to reach the level of systematic uncertainty
reported in Table~\ref{tab:measured_xsec}. This is primarily because systematic uncertainties on this measurement are mainly associated with
background estimates, i.e. processes other than \zinvg, but for $\ETgamma > 600\unit{GeV}$ more than 90\% of the observed monophoton event yield
from 13\unit{TeV} collisions is estimated to come from \zinvg. The HL-LHC~\cite{ref:HLLHC}, scheduled to begin operation in 2026, will collect an anticipated
3000\fbinv\ of 14\unit{TeV} collision data over a ten-year span, which could reduce the percentage statistical uncertainty to one tenth of its
current value, matching the current level of systematic uncertainty.

The expected limit on \mmed\ for the DM simplified models, based on prefit expected yields,
it 200\unit{GeV} higher than the observed limit. This is consistent with the excess of observed data at high \ETgamma\ compared to the prefit
expected yield (see Fig.~\ref{fig:postfitDM_SR}), and also consistent with the larger value of the expected limits on $\Lambda$ and \mD\ compared
to the observed limits (see Figs.~\ref{fig:DMEWKlimits}, \ref{fig:MDLimits}). A total of 16 observed events populate the highest \ETgamma\ bin
in both signal regions combined, and the statistical errors relating to this bin are a sizable fraction of the total event count, motivating the
conjecture that the large ratio of observed to predicted events is a statistical fluke. However, as noted in sec.~\ref{sec:results_znng_xsec}, the probability
of the \zinvg\ process alone contributing the number of observed events in the highest \ETgamma\ bin is rather small, assuming that the \zinvg\ cross
section is adequately modeled by the MATRIX NNLO predictions and that the behavior of \zinvg\ events in the CMS experiment is adequately modeled
by the MadGraph5+GEANT4 simulations. Either of these may in fact be insufficiently accurate at high \pTgamma, relative to current experimental
sensitivity. Alternatively, some unknown or inadequately modeled source of background may be making a significant contribution. The most
provocative possibility is that this represents the first subtle glimpse of BSM physics, though the estimated probability of a \zinvg\ fluctuation
is thousands of times higher than the accepted threshold to claim such a discovery. Some combination of any or all of these possibilities may be true.
Additional data would improve the statistical precision of the high-\ETgamma\ yield estimates, sharpening any potential excesses, or else
``regressing to the mean'' and leveling out to the current SM predictions.

Higher-energy \Pp\Pp\ collisions are a double-edged sword in terms of analysis sensitivity. More high-\ETgamma\ \zinvg\ events would
be produced per collision, but so would more high-\ETgamma\ background events, which would introduce an increased level of systematic uncertainty
in the high \ETgamma\ bins. The LHC is expected to increase its collision energy from 13\unit{TeV} to 14\unit{TeV} within the next two years,
and this modest increase in energy should not dramatically alter the current systematic uncertainties, but the difference may be
noticeable\footnote{The yields from noncollision sources may or may not track the collision energy very smoothly, but in any case these are currently so small
(see Fig.~\ref{fig:postfitDM_SR}) that even a factor of two increase would not significantly impact any final results.}. On the other hand,
in the BSM models considered here, more extreme values of the parameters tend to shift the largest BSM yields to higher energies,
and so higher collision energies will probe and potentially exclude larger swaths of parameter space in these theories. This tendency has been
demonstrated before in the progression from 8\unit{TeV} to 13\unit{TeV} collisions. Using 12.9\fbinv\ of 13\unit{TeV} data collected in the
first half of 2016, compared to 19.6\fbinv\ of data collected at 8\unit{TeV}, and with a similar analysis methodology in each case (e.g. without
the likelihood fits developed in this thesis), the CMS experiment was able to exclude at 95\% CL values of \mD\ up to between 2.31 at low $n$ and 2.49\unit{TeV}
at high $n$~\cite{ref:JHEP10(2017)073}, compared to the 8\unit{TeV} 95\% CL lower limits of between 1.97\unit{TeV} at high $n$ and 2.12\unit{TeV} at low $n$~\cite{ref:j.physletb.2016.01.057}
(see Fig.~\ref{fig:add_cms_earlylimits_MD}).
For the DM EFT considered at 8\unit{TeV}, 95\% CL lower limits on the suppression scale $M_\mathrm{*}$ obtained from the same 12.9\fbinv\ analysis\footnote{Via
the translation $M_\mathrm{*} = \mmed/\sqrt{\gq\gdm}$.} are about twice as high as the 90\% CL lower limits found in the 8\unit{TeV} analysis, for both the vector and
axial-vector cases. The progression from 13\unit{TeV} to 14\unit{TeV} may not be as dramatic, but is still expected to push the BSM sensitivity even further.
