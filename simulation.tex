\chapter{Simulation} \label{chap:simulation}
\section{Parton distribution functions} \label{sec:simulation_pdf}
A hard collision involving a proton can cause individual partons within it to be clearly resolved.
For collisions with characteristic energy $Q$, the probability that the parton will be of type $i$
and carry a fraction $x$ of the proton's total momentum is given by a function $f_{i}(x, Q^{2})$
known as a parton distribution function (PDF).

The dependence of these functions on $Q$ is described by the DGLAP equations~\cite{ref:0370-2693(71)90576-4, ref:dokshitzer, ref:0550-3213(77)90384-4},
which are derived from the theory of QCD.
These allow the specific values of $f_{i}(x, Q^{2})$ to be calculated at arbitrary $Q$ as long as they
are first measured at some particular $Q$.
Like cross sections, the DGLAP equations, and hence PDFs derived in this way, admit perturbative expansions out to specific orders in
$\alpha_\mathrm{S}$~\cite{ref:0034-4885/70/1/R02}.

Simulated samples used in this thesis use PDFs computed to different orders as appropriate for the order
of the cross section used. Samples with cross sections computed to LO in $\alpha_\mathrm{S}$ use the NNPDF3.0 LO PDF
set~\cite{ref:NNPDF30}. Samples with NLO cross sections, as well as theoretical \zinvg, \wlng, and \zllg\ cross sections
computed to NNLO in MATRIX, use the NNPDF3.1 NNLO PDF set~\cite{ref:NNPDF31}.
These PDF sets are based on fits of the theoretical DGLAP functional form to the experimentally-measured values
of the PDFs from multiple experiments, examining a variety of processes at a range of different energy scales.

\section{Hard process generation} \label{sec:simulation_hard_process}
A generic hard proton scattering process proceeds as
\begin{equation}
\Pp + \Pp \to X + rest
\label{eq:proton_interaction}
\end{equation}
where $X$ is a high-energy final product (such as a \Pgamma) of a hard parton--parton collision, and $rest$ is a collection of
lower-energy final-state hadrons emerging from softer parton interactions.

According to the QCD factorization theorem, the dominant contribution to the cross section $\sigma_{X}$ of such a process is given by (see e.g.~\cite{ref:BargerPhillips})
\begin{equation}
\sigma_{X} = \sum_{i,j}\int \! dx_{1}dx_{2} \, f_{i}(x_{1}, \mu_{F}^{2})f_{j}(x_{2}, \mu_{F}^{2})\sigma_{i,j \to X}
\label{eq:qcd_factorization}
\end{equation}
where $i,j$ specify the types of the hard scattering partons, and $\sigma_{i,j \to X}$ is the cross section for the partonic process
\begin{equation}
i + j \to X + rest
\label{eq:parton_interaction}
\end{equation}
which can be computed to a desired order using Feynman diagrams\footnote{Here $rest$ is a collection of quarks and gluons
that eventually fragment into final-state hadrons.}. The energy scale $\mu_{F}$ is called the \textit{factorization scale}.
The value of $\alpha_\mathrm{S}$, which appears in perturbative expansions of $\sigma_{i,j \to X}$, is governed by another energy scale $\mu_{R}$ called
the \textit{renormalization scale}\footnote{As the characteristic
energy scale increases, the effective value of $\alpha_\mathrm{S}$ decreases. This is in contrast to the fine structure ``constant'' $\alpha$,
whose value increases with increasing energy.}.
The coefficients of $\alpha_\mathrm{S}$ in such perturbative expansions also depend on $\mu_{F}$ and $\mu_{R}$,
in such a way as to completely cancel any $\mu_{F}, \mu_{R}$ dependence in the exact formula for the total cross section $\sigma_{X}$~\cite{ref:0034-4885/70/1/R02}.

The values of $\mu_{F}$ and $\mu_{R}$ do, however, affect the approximate cross section derived at a finite order in $\alpha_\mathrm{S}$.
Typically these scales are chosen to be on the order of the parton--parton collision energy.
The simulated samples used in this thesis were made by fixing both $\mu_{F}$ and $\mu_{R}$ to the mass of the \PZ-boson, about 91.2\unit{GeV}.
This choice is a source of theoretical uncertainty, whose magnitude is estimated in this thesis by
examing the variation of the final results when $\mu_{F}$ and $\mu_{R}$ are separately shifted up and down\footnote{Following common practice, correlated shifts up and down
are also considered, but ``unphysical'' anticorrelated shifts (e.g. $2{*}\mu_{F}, 0.5{*}\mu_{R}$) are not~\cite{ref:scalepdf_bendavid}.} by a factor of 2.

Differentiating $\sigma_{X}$ with respect to a kinematic variable such as \pTgamma\ yields $d\sigma_{X}/d\pTgamma$, the differential
cross section for the process $\Pp + \Pp \to X + rest$ to occur with some specific final-state \pTgamma.
For an elementary process like $i + j \to X + rest$, this may be analytically solved or numerically approximated out to a given order.

However, the full dynamics of a \Pp\Pp\ interaction and subsequent interactions of the products with the CMS detector do not admit simple solutions.
These and other integrals over probability distributions may be estimated
via the \textit{Monte Carlo} (MC) method. Many random events\footnote{``Monte Carlo'' alludes to an eponymous casino in Monaco.}
are generated according to known probability distributions,
and the integrand (e.g. some reconstructed kinematic quantity) is evaluated for each event.
The sum of the integrand values for each event within a given kinematic region,
divided by the total number of events generated, gives an estimate of the integral within that region (e.g. the
content of a bin in a histogram).

Matrix element (ME) generators allow for MC estimates of $d\sigma_{X}/d\pTgamma$ by generating
partons with species and initial-state kinematics sampled according to the PDFs, and with final-state kinematics sampled
according to the differential partonic cross sections $d\sigma_{i,j \to X}/d\pTgamma$.
The ME generator for all simulated samples used in this thesis is MadGraph5\_aMC@NLO~\cite{ref:JHEP07(2014)079},
which also constructs the matrix elements for the elementary partonic processes.
All\footnote{Aside from a small subset of DM simplified model samples generated to NLO.} samples are generated to LO in $\alpha_\mathrm{S}$.
Since \zinvg, \wlng, and \zllg\ are all generated to LO, their simulation does
not incorporate Feynman diagrams with extra radiated quarks and gluons. These extra particles are known
to appear at higher orders, and since they can substantially affect the kinematic balance and reconstruction
efficiency of events, up to 2 of these are added by the generator to simulate their impact.

If it is decided that the probability of an event's occurrence should be changed by some multiplicative factor $w$, this can be simulated
by multiplying a generated event's weight by $w$ in any final sums. By this principle, uncertainties associated with the PDFs are
represented in the NNPDF sets by 100 different weights on each event~\cite{ref:NNPDF30,ref:NNPDF31}. The PDF uncertainty of a histogram
estimated via MC is obtained by taking the sample standard deviation of each bin in the histogram with respect to these 100 weight variations~\cite{ref:0954-3899/43/2/023001}.

The NNLO cross sections for \zinvg, \wlng, and \zllg\ are estimated using the MATRIX generator~\cite{ref:epjc/s10052-018-5771-7},
which incorporates NNLO ME results for \PZ\Pgamma\ and \PW\Pgamma\ processes derived in~\cite{ref:j.physletb.2014.02.037, ref:JHEP07(2015)085, ref:JHEP12(2015)047}.
An LO-to-NNLO event weight is then applied to each event simulated in MadGraph, by taking the NNLO-over-LO ratio of the differential cross section $d\sigma/d\pTgamma$
integrated within a \pTgamma\ bin corresponding to the generated \Pgamma.
An additional \pTgamma-dependent weight that accounts for NLO EWK corrections, computed in~\cite{ref:JHEP04(2015)018, ref:JHEP02(2016)057}, is also applied.
The impact of these corrections is illustrated in Fig.~\ref{fig:NLO_EWK_weights}.

\begin{figure}[hbtp]
  \begin{center}
    \includegraphics[width=0.49\textwidth]{Figures/Zg_NLO_weights.png}
    \includegraphics[width=0.49\textwidth]{Figures/Wg_NLO_weights.png}
    \caption{
    Fractional shifts on event weights introduced by NLO EWK corrections, as a function of \pTgamma. Left: \zinvg. Right: \wlng. From~\cite{ref:JHEP04(2015)018, ref:JHEP02(2016)057}.
    }
    \label{fig:NLO_EWK_weights}
  \end{center}
\end{figure}

To avoid numerical divergences that can arise in perturbative calculations, and also to avoid wasting computational cycles simulating events outside the analysis scope of this
thesis, kinematic cuts are applied on the events that may be generated. Most significantly, BSM and $\PW,\PZ + \Pgamma$ events are only simulated if they have a leading
\Pgamma\ with $\pT > 130\unit{GeV}$ emitted within the ECAL acceptance. For consistency, the same generator-level cuts (as well as PDF sets) are applied in both MATRIX and MadGraph.

\section{Parton showering and hadronization} \label{sec:simulation_parton_shower_hadronization}
The output of the hard collision, $X + rest$, continues to evolve as it exits the interaction point.
For example, high-energy quarks and gluons will tend to radiate other quarks and gluons in a cascading process
known as a \textit{parton shower}, dispersing their energy as they do so. The energies of the showering partons begin at the hard interaction energy scale
and go down to the hadronization scale (${\lesssim}10\unit{GeV}$). Above this energy scale, $\alpha_\mathrm{S}$ is less than $O(1)$, and so the parton
shower process admits a perturbative analysis in the theory of QCD.
The statistical behavior of parton showering is governed by Sudakov form factors derived from the DGLAP equations~\cite{ref:0034-4885/70/1/R02}.

Below the hadronization scale, the remnant partons will no longer have enough energy to freely split and will finally bind into hadrons
in a process called hadronization. This occurs as the value of $\alpha_\mathrm{S}$ associated with a typical parton energy rises above $O(1)$, rendering estimates
based on perturbative expansions invalid. The bulk of the partons in the colliding protons are not directly involved in the hard parton--parton scatter and are also subject
to a nonperturbative energy scale; these contribute softer showers of particles\footnote{The $rest$ terms in the equations of sec.~\ref{sec:simulation_hard_process}.}
known as the underlying event.
These nonperturbative processes are simulated with generator-dependent phenomenological models, tuned to match the hadronic distributions observed in the experiment.
For all the simulated samples used in this thesis, both parton showering and hadronization are handled by PYTHIA 8.218~\cite{ref:j.cpc.2015.01.024},
with underlying-event tune CUETP8M1~\cite{ref:epjc/s10052-016-3988-x}.

\section{Pileup and detector simulation} \label{sec:simulation_detector}
The simulated contribution from pileup is estimated by overlaying additional ``minimum-bias'' events, with simulated particle distributions
corresponding to the typical \Pp\Pp\ scattering events most likely to occur in the LHC. These are estimated by taking LHC collision data with no
special trigger selections applied. This step of the simulation is also handled by PYTHIA.

The output of the preceding steps is a collection of several thousand stable or semistable particles that continue to propagate outward and interact
with the detector. This stage is handled by GEANT4~\cite{ref:S0168-9002(03)01368-8}, which simulates the decay of long-lived
($0.01\unit{m}/c \lesssim \mathrm{lifetime} \lesssim 10\unit{m}/c$) particles within the detector volume, miscellaneous physical processes including those
discussed in the previous chapter (magnetic bending, electromagnetic showers, hadronic showers, scintillation, etc.), and the response of detector electronics,
leading to the final digitized detector output.
