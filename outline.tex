%%%%%%%%%%%%%%%%%%%%%%%%%%%%%%%%%%%%%%%%%%%%%%%%%%%%%%%%%%%%%%%%%%%%%%%%%%%%%
%%%% Preamble
%%%%%%%%%%%%%%%%%%%%%%%%%%%%%%%%%%%%%%%%%%%%%%%%%%%%%%%%%%%%%%%%%%%%%%%%%%%%%

%%%% The uwthesis.sty file relies on the memoir class!
%%%% You should be using the memoir class anyway; it makes life easier:
%%%% http://www.ctan.org/tex-archive/macros/latex/contrib/memoir/
\documentclass[oneside, letterpaper, 12pt, oldfontcommands]{memoir}

%%%% Import uwthesis.sty to get official formatting, then set your variables.
\usepackage{uwthesis}
% \usepackage{hyperref}
% \usepackage{cite}
% \usepackage{graphicx}
\newcommand{\unit}[1]{\ensuremath{\,\si{#1}}}

\newcommand{\fbinv}{\ensuremath{fb^{-1}}}

% Particles and fields
\newcommand{\PZ}{\ensuremath{Z}}
\newcommand{\PW}{\ensuremath{W}}
\newcommand{\PV}{\ensuremath{V}}
\newcommand{\Pgamma}{\ensuremath{\gamma}}
\newcommand{\Pn}{\ensuremath{\nu}}
\newcommand{\Pan}{\ensuremath{\bar{\Pn}}}
\newcommand{\PWplus}{\ensuremath{W^\mathrm{+}}}
\newcommand{\PWminus}{\ensuremath{W^\mathrm{-}}}
\newcommand{\PWone}{\ensuremath{W^{1}}}
\newcommand{\PWtwo}{\ensuremath{W^{2}}}
\newcommand{\PWthree}{\ensuremath{W^{3}}}
\newcommand{\PB}{\ensuremath{B}}
\newcommand{\Pg}{\ensuremath{g}}
\newcommand{\PH}{\ensuremath{H}}
\newcommand{\Pp}{\ensuremath{p}}
\newcommand{\Pq}{\ensuremath{q}}
\newcommand{\Paq}{\ensuremath{\bar{q}}}
\newcommand{\Pu}{\ensuremath{u}}
\newcommand{\Pd}{\ensuremath{d}}
\newcommand{\Pc}{\ensuremath{c}}
\newcommand{\Ps}{\ensuremath{s}}
\newcommand{\Pt}{\ensuremath{t}}
\newcommand{\Pb}{\ensuremath{b}}
\newcommand{\Pe}{\ensuremath{e}}
\newcommand{\Pmu}{\ensuremath{\mu}}
\newcommand{\Ptau}{\ensuremath{\tau}}
\newcommand{\Pne}{\ensuremath{\Pn_\mathrm{e}}}
\newcommand{\Pnmu}{\ensuremath{\Pn_\mathrm{\mu}}}
\newcommand{\Pntau}{\ensuremath{\Pn_\mathrm{\tau}}}
% \newcommand{\ell}{\ensuremath{l}}

% Processes
\newcommand{\zinvg}{\ensuremath{\PZ(\Pn\Pan)\Pgamma}}
\newcommand{\wlng}{\ensuremath{\PW(\ell\Pn)\Pgamma}}
\newcommand{\zllg}{\ensuremath{\PZ(\ell\ell)\Pgamma}}
\newcommand{\gjets}{\ensuremath{\Pgamma{+}\mathrm{jets}}}

% Kinematic variables
\newcommand{\pT}{\ensuremath{p_\mathrm{T}}}
\newcommand{\pTgamma}{\ensuremath{p^{\gamma}_\mathrm{T}}}
\newcommand{\vecpT}{\ensuremath{\vec{p}_\mathrm{T}}}
\newcommand{\vecpTgamma}{\ensuremath{\vec{p}^{\,\gamma}_\mathrm{T}}}
\newcommand{\ET}{\ensuremath{E_\mathrm{T}}}
\newcommand{\ETgamma}{\ensuremath{E_\mathrm{T}^{\gamma}}}
\newcommand{\MT}{\ensuremath{M_\mathrm{T}}}
\newcommand{\MET}{\ensuremath{p_\mathrm{T}^\mathrm{miss}}}
\newcommand{\vecMET}{\ensuremath{\vec{p}_\mathrm{T}^\mathrm{\,miss}}}
\newcommand{\vecrecoil}{\ensuremath{\vec{U}}}
\newcommand{\sieie}{\ensuremath{\sigma_{i\eta i\eta}}}
\newcommand{\sipip}{\ensuremath{\sigma_{i\phi i\phi}}}

% ML fit parameters
\newcommand{\nZg}[1][]{\ensuremath{n^{\PZ\Pgamma}_{#1}}}
\newcommand{\nWg}[1][]{\ensuremath{n^{\PW\Pgamma}_{#1}}}
\newcommand{\RZll}[1][]{\ensuremath{R^{\PZ\Pgamma}_{\ell\ell\Pgamma#1}}}
\newcommand{\RZee}[1][]{\ensuremath{R^{\PZ\Pgamma}_{\Pe\Pe\Pgamma#1}}}
\newcommand{\RZmm}[1][]{\ensuremath{R^{\PZ\Pgamma}_{\Pmu\Pmu\Pgamma#1}}}
\newcommand{\RWl}[1][]{\ensuremath{R^{\PW\Pgamma}_{\ell\Pgamma#1}}}
\newcommand{\RWe}[1][]{\ensuremath{R^{\PW\Pgamma}_{\Pe\Pgamma#1}}}
\newcommand{\RWm}[1][]{\ensuremath{R^{\PW\Pgamma}_{\Pmu\Pgamma#1}}}
\newcommand{\fZW}[1][]{\ensuremath{f^{\PZ\Pgamma}_{\PW\Pgamma#1}}}
\newcommand{\nhalo}[1][]{\ensuremath{n^{\text{halo}}_{S#1}}}
\newcommand{\nll}[1][]{\ensuremath{n_{\ell\ell\Pgamma#1}}}
\newcommand{\nl}[1][]{\ensuremath{n_{\ell\Pgamma#1}}}
\newcommand{\nS}[1][]{\ensuremath{n_{S#1}}}

% Hyphenation of uncommon terms
\hyphenation{cal-ori-me-ter}
\hyphenation{cal-ori-me-ters}

\settitle{A Measurement of \zinvg\ Production and a Search for New Physics in
Monophoton Events Using the CMS Detector at the LHC}
\setauthor{James Joseph Buchanan}
\setdepartment{Physics}
\doctors % or \masters
\setgraddate{2018}
\setdefensedate{T.B.D.} % or whatever format you want

%%%% Members of the Final Oral Committee (FOC)
%%%% Give name, rank, and department
%%%% 
\setfoca{Sridhara Dasu}{Professor}{Physics} % <- Your advisor
\setfocb{Wesley Smith}{Emeritus Professor}{Physics}
\setfocc{Matt Herndon}{Professor}{Physics}
\setfocd{T.B.D.}{Professor}{Physics}
\setfoce{T.B.D.}{Professor}{Something Else}
% \setfocf{Grover Cleveland}{Professor}{Zoology}

%%%% Your abstract, used for the UMI abstract and in your front matter
\setabstract{%
  This thesis presents several studies of monophoton final states
  using 35.9 $fb^{-1}$ of 13 TeV proton-proton collision data collected by the CMS
  experiment at the LHC in 2016. The standard model \zinvg\ cross section is measured
  as a function of photon transverse momentum. No significant deviations from standard
  model predictions are observed.
  The results are also interpreted in the context of several new physics models.
  Limits are placed on coupling strengths of anomalous triple gauge couplings between
  photons and \PZ\ bosons,
  new particle masses in simplified models of dark matter, the suppression scale of a dark matter
  effective field theory model, and the graviton mass scale in a model of extra
  spatial dimensions.
}

%%%%%%%%%%%%%%%%%%%%%%%%%%%%%%%%%%%%%%%%%%%%%%%%%%%%%%%%%%%%%%%%%%%%%%%%%%%%%
%%%% Document
%%%%%%%%%%%%%%%%%%%%%%%%%%%%%%%%%%%%%%%%%%%%%%%%%%%%%%%%%%%%%%%%%%%%%%%%%%%%%

\begin{document}

% Tell the memoir class to set up lowercase roman for pagination, etc.
\frontmatter

%%%% Uncomment this to create a UMI abstract page.
%%%% If you are submitting electronically, however, this page is unnecessary.
% \theumiabstract

% The title page
\thetitlepage
\clearpage

% The copyright page, if you want to pay the fee and register copyright.
% \thecopyrightpage
% \cleardoublepage

% These above pages should not be counted, so we reset the counter to 1.
\setcounter{page}{1}

% An abstract may be required by your department.
\section{Abstract}
\uwabstract
\cleardoublepage

% Acknowledgements go here if you want to include them.
% \section{Acknowledgements}
% Acknowledgements go here.
% % These results would never have come to fruition without the support of my colleagues
% % and friends in the UW-Madison CMS group.
% % Wesley Smith, Sridhara Dasu, and Matt Herndon are great physicists and leaders
% % who have nurtured a rigorous standard of excellence that this group
% % can rightly be proud of.
% % Bhawna Gomber's deep expertise, organizational acumen,
% % and untiring work ethic have pushed these analyses forward at every step. These
% % results are as much hers as they are mine.
% % Tom Perry developed a previous iteration of these analyses,
% % and freely contributed his assistance and code to let us hit the ground running in 2016.
% % Devin Taylor, Tyler Ruggles, Nate Woods, Laura Dodd, Nick Smith, Kenneth Long, and
% % Usama Hussein were all excellent office mates, always available to lend
% % their considerable knowledge and insight at the drop of a hat, a privilege I have taken advantage
% % of on innumerable occasions.
% % I must finally thank my parents, Yvonne and Darryl Buchanan,
% % for their perpetual support of everything I do.
% \clearpage

% Table of contents
% \maxtocdepth{subsection}
% \tableofcontents* % the * means that there isn't an entry for the TOC itself
% \clearpage
% \listoffigures  % if you have any figures
% \clearpage
% \listoftables   % if you have any tables

% Tell the memoir class to set up normal pagination, etc. for the main doc
\mainmatter

\chapter{Introduction} \label{sec:introduction}
\section{Overview} \label{sec:introduction_overview}
This thesis presents several analyses of event
yields in ``monophoton'' final states, characterized by a single \Pgamma\ with high transverse
momentum, along with an overall transverse momentum imbalance typically of equal magnitude and opposite direction to
that of the photon.
These analyses correspond to 35.9 \fbinv\ of 13 TeV proton-proton (\Pp\Pp) collision data collected in 2016 by the CMS
detector at the LHC. A measurement of the production rate for the process $\Pp\Pp \to \PZ\Pgamma \to \Pn\Pan\Pgamma$ is obtained
and compared to predictions derived from the standard model (SM) of particle physics. No significant deviation from SM
predictions is observed.

The predicted monophoton yield in several theories of physics beyond the SM (BSM) is higher than the SM prediction.
This thesis examines two varieties of anomalous triple gauge coupling (aTGC), simplified models of dark matter (DM)
interacting with SM matter via a vector or axial-vector mediator, an effective field theory (EFT) of DM interaction
with  \Pgamma\ and \PZ\ bosons, and a model of extra spatial dimensions. For each of these models, 95\% confidence level (CL)
limits are placed on relevant parameters based on the observed collision data.

\section{Standard model of particle physics} \label{sec:introduction_standard_model}
Introduce basic SM concepts and vocabulary.

\section{\zinvg\ cross section} \label{sec:introduction_znng}
Introduce the \zinvg\ process and discuss its cross section. Include a discussion of NNLO QCD and NLO EWK corrections.
\subsection{Previous measurements}
List previous experimental work constraining the \zinvg\ cross section.

\section{Anomalous triple gauge couplings} \label{sec:introduction_aTGC}
Introduce the effective vertex model that parametrizes our aTGC limits.
\subsection{Previous searches} \label{sec:introduction_aTGC_previous_searches}
List previous experimental work constraining the \PZ-\Pgamma\ aTGC parameters via the study of \PZ\Pgamma\ processes.

\section{Dark matter simplified models} \label{sec:introduction_dark_matter}
A brief introduction to DM followed by an introduction to the DM simplified models with a vector and axial-vector
mediator, as well as the EWK EFT model describing a direct DM-EWK boson interaction.
\subsection{Previous searches}
List previous experimental work constraining the model parameters, with a focus on the monophoton channel.

\section{ADD gravitons} \label{sec:introduction_ADD}
Introduce the ADD extra dimensions model and the phenomenological signature of ADD graviton emission.
\subsection{Previous searches}
List previous experimental work constraining ADD model parameters, with a focus on the monophoton channel.

\chapter{The CMS experiment and the LHC}
\section{The LHC}
\subsection{Proton acceleration}
\subsection{Magnets and beam halo}
Beam halo is an important component of the monophoton analysis and it comes from here.
\section{The CMS experiment}
\subsection{Coordinate system}
\subsection{Superconducting solenoid and silicon tracking system}
\subsection{Electromagnetic calorimeter}
The discussion of APDs segues into an introduction to ECAL spikes.
\subsection{Hadronic calorimeter}
\subsection{Muon systems}
\subsection{Trigger system}

\chapter{Simulation}
\section{Hard process generation}
\section{Parton distribution functions}
\section{Parton showering and hadronization}
\section{Pileup simulation}
\section{Detector simulation}

\chapter{Object reconstruction and selection}
\section{The particle-flow algorithm}
\section{Photons and electrons}
\section{Muons}
\section{Jets and missing transverse momentum}
A discussion of jets is followed by a definition of \MET\ and Type-1 \MET\ corrections.
\section{Primary vertex}

\chapter{Event selection}
\section{Backgrounds}
List the sources of background in the monophoton channel, to justify the ensuing cuts.
\section{Trigger and \MET\ filters}
Trigger path, trigger efficiency

\MET\ filters
\section{Photon}
Photon kinematic cuts; Photon ID defintion, efficiency; Spike and beam halo cuts; phoET-dependent cross section corrections
\section{Missing transverse momentum}
$\MET > 170$ GeV; $\Delta\phi(\Pgamma,\vecMET) > 0.5$; $\mathrm{min}\Delta\phi(\mathrm{jets},\vecMET) > 0.5$;
$\ETgamma/\MET < 1.4$
\section{Lepton vetoes}
Electron selection; Muon selection
\section{Single electron control region}
\section{Single muon control region}
\section{Dielectron control region}
\section{Dimuon control region}

\chapter{Background estimation}
For each component, describe its estimation and uncertainties
\section{Simulated backgrounds}
\section{Electron faking photon}
\section{Jet faking photon}
\section{Spikes}
\section{Beam halo}
\section{Transfer factors}
\section{Likelihood function}

\chapter{Results}
\section{\zinvg\ cross section}
\section{aTGC limits}
\section{DM simplified model limits}
\section{ADD limits}

\chapter{Conclusions}
\section{Summary}
\section{Outlook}

% \bibliographystyle{utcaps}
% \bibliography{references}
\end{document}
