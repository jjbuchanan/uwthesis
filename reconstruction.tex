\chapter{Object reconstruction} \label{chap:reconstruction}
\section{The particle-flow algorithm} \label{sec:reconstruction_particle_flow}
CMS uses the \textit{particle-flow} (PF) algorithm to attempt to reconstruct the identities and kinematic properties of all final-state particles in
a collision event. CMS is the first hadron collider experiment to have successfully implemented this strategy,
owing to its precise and robust tracking systems, highly granular and hermetic calorimeters, and intense solenoidal magnetic field.
The PF algorithm is described in detail in~\cite{ref:1748-0221/12/10/P10003}.

Particles with lifetimes long enough to exit the beam pipe and reach CMS are divided into six general classes:
electrons, photons, charged hadrons, neutral hadrons, muons, and neutrinos.
Neutrinos do not interact with any of the detector layers, and so PF only attempts to directly reconstruct the other five classes.
Electrons, charged hadrons, and muons leave tracks in the inner tracker, while photons and neutral hadrons do not. Photons and electrons
deposit nearly all their energy in electromagnetic showers in ECAL, which completely stops them from propagating forward to subsequent
detector layers. Generally speaking, hadrons and muons do not produce substantial ECAL showers, and these particles continue on to HCAL.
Hadronic showers usually begin in one of the calorimeters and evolve mainly in HCAL, which fully absorbs them before they can reach the muon detectors.
Muons do not produce hadronic showers and, being MIPs, pass through the inner tracker, calorimeters, solenoid, steel flux-return yoke,
and muon detectors largely unperturbed aside from magnetic bending, leaving tracks in the muon detectors on their way out.
These interactions are illustrated in Fig.~\ref{fig:particle_flow}.

\begin{figure}[hbtp]
  \begin{center}
    \includegraphics[width=0.99\textwidth]{Figures/particle_flow.png}
    \caption{
    A schematic diagram of CMS overlaid with
    the classes of particles reconstructed by the PF algorithm, showing their most
    characteristic interactions with detector layers.
    From~\cite{ref:1748-0221/12/10/P10003}.
    }
    \label{fig:particle_flow}
  \end{center}
\end{figure}

Each detector layer gives rise to basic characteristic elements used in event reconstruction:
tracks from the inner and muon trackers, and clusters from the ECAL and HCAL. The PF algorithm links these elements together
so as to deduce the species of a particle based on the set of elements present along its trajectory. The linked elements
corresponding to a given particle then provide multiple pieces of data on the particle's behavior, which are synthesized
in order to refine the measurement of the particle's kinematic properties.

Like the physical detector and the trigger apparatus, the reconstruction algorithms employed by CMS are subject to ongoing evolution
and optimization. The details given in this chapter correspond to the specific strategies used in 2016.

\section{Tracks and clusters}
Track reconstruction in the inner tracker is performed over 10 iterations, with the parameters of each iteration tuned to maintain a high selection purity
and reconstruction speed. In each iteration, seeds comprising a few tracker hits are used to initiate a Kalman filter (KF) track reconstruction algorithm, which assembles
the tracks in steps one tracker layer at a time.
A partially-assembled track is extrapolated to a subsequent layer to identify a compatible hit to add, then a new overall track is fit to the updated series
of compatible hits, and the process is repeated until every tracker layer is exhausted~\cite{ref:cms_tdr_vol1}.
Hits assigned to tracks in one reconstruction iteration are then ignored (\textit{masked}) in successive iterations so as to reduce the probability of a misreconstructed track,
and so track quality criteria can be loosened in each iteration to raise the tracking efficiency while simultaneously keeping the purity high.

Clustering is performed separately in EB, EE, HB, HE, and each of the two preshower layers on either endcap\footnote{Separate cells are not clustered in HF.
Rather, each individual HF cell gives rise to one HF EM cluster and one HF HAD cluster, corresponding
to the electromagnetic and hadronic components of each cell, respectively.}. The algorithm proceeds in the same general
fashion in each of these components, with distinct component-dependent threshold values~\cite{ref:1748-0221/12/10/P10003} applied at various steps.
Since this thesis focuses on barrel photons, only EB thresholds are reported here.

An EB crystal is identified as a cluster seed if its energy exceeds 230\unit{MeV} as well as that of its 8 immediate neighbors. Neighboring cells
are progressively added to a topological cluster based on the seed, as long as their energy exceeds 80\unit{MeV} (twice the average noise level in EB)
and they share at least one corner in common with a cell in the cluster. A single topological cluster can grow to encompass multiple seeds.
Once a topological cluster is fully built, the energy distribution in the topological cluster is fit to a sum of Gaussian shapes,
with one Gaussian of width $\sigma = 1.5\unit{cm}$ per seed. Each of the fit shapes then becomes its own cluster, with a position and energy defined by the Gaussian's
mean and amplitude, respectively. In this manner, the energy of a single crystal can be shared between distinct clusters.

The response of calorimeter cells is periodically calibrated so as to faithfully reproduce the shower energy they contain.
Due to the thresholds imposed during clustering, some of this energy will tend to be left out of PF clusters, and therefore another
layer of calibrations is applied to these clusters in order to restore this energy\footnote{At this stage, the calibrated energy of a cluster in the preshower detector
is added to the calibrated energy of an EE cluster to give a single EE cluster energy.}.

A calorimeter cluster does not have a well-defined momentum, but it does have an energy and $\theta$ coordinate. This makes it possible to define the \textit{transverse
energy} \ET\ of a cluster, which is energy times $\sin{\theta}$.
The \ET\ of a highly-relativistic particle is approximately equal to its \pT, and exactly equal if the particle is massless. In particular, if a cluster contains the
full energy of a photon and has an accurately-reconstructed $\theta$ coordinate, then the photon's \pT\ is the same as the cluster \ET.

A linking algorithm assigns a link between two PF elements if they overlap sufficiently closely in the $\eta$--$\phi$ plane.
Linked elements are then chained together into lattices called blocks. A block corresponding to a simple isolated particle trajectory
may have a simple linear topology, but generic blocks can have treelike structures with various converging and diverging lines.
This reflects the tendency of particles to radiate or decay into other
particles, as well as the possibility that multiple particles may be collimated so closely that their separate contributions to PF elements cannot
reliably be distinguished by just the linking algorithm. Subsequent stages of the PF algorithm identify the particle(s) corresponding to each
block. The most robustly-reconstructed particles are identified first and have their contributions to PF elements
subtracted out from those elements, leaving behind residual excesses which can then be identified as other particles.

\section{Muons} \label{sec:reconstruction_muons}
A dedicated muon track reconstruction algorithm is seeded by hits in the DTs and CSCs. Hits in the DTs, CSCs, and RPCs are linked along a potential muon trajectory,
resulting in a \textit{standalone-muon} track.
A track in the inner tracker qualifies as a \textit{tracker-muon} track if its extrapolated trajectory matches at least one hit in the muon systems.
If an inner track fits together with a standalone-muon track, hits in both tracks are fit to a common \textit{global-muon} track.
About 99\% of muons produced within the $\eta$ coverage of the muon systems are reconstructed as a global or tracker muon\footnote{They are often reconstructed
as both. A separate global muon and tracker muon are merged into a single muon candidate if they share the same inner track.}.
The additional hit data significantly improves the momentum resolution of muons with $\pT \gtrsim 200\unit{GeV}$~\cite{ref:1748-0221/12/10/P10003}.

A PF muon candidate is retained if the tracks and calorimeter deposits within $\Delta R < 0.3$ of the candidate muon have a total \pT\ no greater than 10\%
of the candidate muon \pT. A muon candidate is also retained if it passes a high-purity ``tight'' muon ID, as defined in~\cite{ref:1748-0221/7/10/P10002},
or if it has a high-quality global or inner track with a large number of hits, as long as any associated calorimeter clusters are compatible with the muon hypothesis.
Muons identified at this stage are the first particles identified by the PF algorithm, and their corresponding PF elements are masked in subsequent steps.
Muons can also be identified later, after charged hadron candidates have been assigned, if a track linked to a charged hadron has a momentum significantly greater than
the sum of the linked calorimeter cluster energies. Once a muon is accepted, its momentum is derived from the orientation and curvature of its track.

\section{Photons and electrons} \label{sec:reconstruction_egamma}
Just prior to the installation of ECAL into CMS, the energy resolution of single electrons in the barrel region was estimated
by firing an electron beam directly at the crystal array~\cite{ref:1748-0221/2/04/P04004}.
These measurements show that a $5{\times}5$ crystal array contains about 97\% of an electron shower's energy.
A fit of that data to the functional form\footnote{The symbol ${\oplus}$ denotes addition in quadrature.}
\begin{equation}
\frac{\sigma}{E} = \frac{S}{\sqrt{E}} \oplus \frac{N}{E} \oplus C
\label{eq:calo_resolution}
\end{equation}
gives $S = 0.028\unit{GeV^{1/2}}$, $N = 0.12\unit{GeV}$, and $C = 0.003$. This indicates that, using
only ECAL shower information, an isolated \Pe\ or unconverted \Pgamma\ with energy around 200\unit{GeV} will typically have a measured energy within 0.4\% of the true value.

This straightforward picture of a single \Pe\ or \Pgamma\ hitting the ECAL is complicated by the significant depth of tracking material in front of it.
As shown in Fig.~\ref{fig:tracker_thickness}, a minimum of ${\approx}0.4$\,$X_{0}$ of material sits in front of EB, rising up to ${\approx}2$\,$X_{0}$ near $|\eta| = 1.4$.
On average, electrons will radiate away about 33\% of their energy through bremsstrahlung near $|\eta| = 0$, and about 86\% of their energy near $|\eta| = 1.4$~\cite{ref:1748-0221/10/06/P06005}.
The electron trajectory is bent in the $\phi$ coordinate by the solenoid's magnetic field, but the radiated photon trajectories are not, so they are spread widely in $\phi$.
Meanwhile, photons have a substantial chance of converting to an $\Pe^{+}\Pe^{-}$ pair in the tracker. The PF algorithm incorporates a dedicated conversion
finder~\cite{ref:1748-0221/10/08/P08010} to identify these cases. The construction of topological clusters in ECAL, also known as \textit{superclusters}, is tuned
to gather as reliably as possible the energy deposits of multiple radiated or converted particles stemming from a common originating \Pe\ or \Pgamma.

\begin{figure}[hbtp]
  \begin{center}
    \includegraphics[width=0.90\textwidth]{Figures/tracker_thickness.png}
    \caption{
    Thickness of material in front of ECAL, in multiples of $X_{0}$ (left) and $\lambda_{I}$ (right). From~\cite{ref:1748-0221/9/10/P10009}.
    }
    \label{fig:tracker_thickness}
  \end{center}
\end{figure}

The KF method used in standard track reconstruction is designed to identify trajectories in steps, accommodating a Gaussian-distributed
jitter in the energy of a particle at every step. However, the amount of energy lost by bremsstrahlung is described not by a Gaussian, but rather a Bethe-Heitler distribution~\cite{ref:rspa.1934.0140}.
This can be approximated as a sum of several Gaussians, and the PF algorithm implements a specially-designed Gaussian-sum filter (GSF)~\cite{ref:0954-3899/31/9/N01}
for electron track reconstruction.

A full GSF electron track is reconstructed if the existence of a GSF track seed is first established.
A standard KF inner track produces a GSF track seed if its \pT\ exceeds 2\unit{GeV}, and either
the track momentum closely matches a compatible ECAL cluster's energy, or else the track has a number of hits, fit quality, and ECAL cluster compatibility
identified as electron-like by a boosted-decision-tree (BDT) classifier. A seed is also identified when an ECAL supercluster with $\ET > 4\unit{GeV}$
can be extrapolated to tracker hits consistent with an electron trajectory.

Bremsstrahlung photons tend to emerge nearly parallel to the parent electron's motion, and so have momentum vectors lying tangent to the electron's track.
As part of the linking algorithm, tangent lines from each GSF track are drawn at each of its intersections with the tracker layers. If one of these tangent lines,
extrapolated forward, is compatible with the position of an ECAL cluster (or a conversion track pair identified by the conversion finder), the cluster (or track pair)
is linked to the track as a possible bremsstrahlung photon.

A GSF electron track goes on to seed a PF electron candidate if the ECAL cluster compatible with the overall track trajetory has no more than two other tracks linked to it.
Additionally, if the GSF track was seeded by an ECAL supercluster, the sum $H$ of the HCAL cell energies within $\Delta R < 0.15$ of the supercluster position,
divided by the supercluster energy $E$, must not exceed 0.10.
An ECAL supercluster seeds an isolated PF photon candidate if its \ET\ is greater than 10\unit{GeV}, it has no links to a GSF track, and $H/E < 0.10$.

The \Pe\ or isolated \Pgamma\ candidate is then formed starting with its candidate seed. In order to gather bremsstrahlung radiation and photon conversion products,
all ECAL clusters linked to either the supercluster or GSF track are then associated to the candidate.
If any of those clusters is linked to an inner track, that track is also associated to the candidate,
as long as the track momentum and the energy of the HCAL cluster linked to it (if one exists) are consistent with the hypothesis that an electron produced the ECAL cluster.
Finally, any remaining tracks and clusters identified by the conversion finder as belonging to a converted photon are associated to a candidate electron if they are linked
to the candidate's GSF track.

A ``Swiss cross'' cut is applied in order to reject spikes:
defining $E_{1}$ to be the energy of the seed crystal and $E_{4}$ to be the sum of the energies of the 4 neighboring ECAL crystals,
a PF photon is only retained if $E_{4}/E_{1} > 0.05$. Double spikes are suppressed by a similar cut based on the ratio of $E_{2}$, the sum of the energies
of the seed crystal and its highest-energy neighbor, and $E_{6}$, the sum of the energies of the six neighboring crystals surrounding the first
two. Further spike rejection is achieved by requiring the time of energy deposits above 1\unit{GeV} to be within 2\unit{ns} of the bunch crossing time.

In order to keep the PF elements of very low-quality candidates available for potential reconstruction into other particles, a fully-assembled \Pe\ or \Pgamma\ candidate
must pass some further loose quality criteria in order to be retained. All the tracks and clusters associated to a retained candidate are then
masked against further PF processing. Electrons and isolated photons kept at this stage are the second set of particles identified by the PF algorithm, after muons.
The direction of a PF photon is assigned to be that of its supercluster, while the direction of a PF electron is that of its GSF track.

Any remaining ECAL clusters in the tracker acceptance ($|\eta| < 2.5$) become nonisolated PF photons, along with any remaining ECAL clusters outside the tracker
acceptance without a link to an HCAL cluster\footnote{All HF EM clusters become HF photons.}.
Nonisolated PF photons can also be formed out of an ECAL cluster energy excess in a charged hadron candidate, as described in sec.~\ref{sec:reconstruction_jetmet}.
As the name implies, nonisolated photons tend to be surrounded by significant energy deposits from nearby particles, which can interfere with their reconstruction quality.

Bremsstrahlung and photon conversions introduce multiple opportunities for energy to be lost as particles escape
through small cracks in the detector, fall into dead regions, or otherwise fail to be associated to the main PF candidate. In order to recover this energy,
a set of particle-specific calibrations is derived for electrons and photons. The energy most commonly reported in CMS analyses is this final calibrated value.

Isolated high-\pT\ photons typically have their energies raised by up to a few percent by the last set of calibrations. However, after examining monophoton events with
probable non-photon objects misreconstructed as photons, we have determined that these ``fake'' photons are often assigned very large energy shifts
on the order of 100\%, badly skewing their energies upward. This is due to the fact that fake photons tend to have atypical photon shower shapes,
and the calibrations, based on simulated ``real'' photon events, were derived as a function of shower shape\footnote{For this reason, calibrations
derived in 2017 and 2018 are independent of shower shape~\cite{ref:egammaclusterregression_2018}.}.
High-\pTgamma\ events have the least precisely measured cross sections, and are also the most sensitive probes of the BSM models examined in this thesis.
In order to minimize contamination of this sensitive kinematic region and thereby maximize the statistical power of our results,
we have opted to use the uncalibrated \textit{raw} photon supercluster \ET\ as the reconstructed quantity in which to bin our data.

\section{Hadrons, jets, and missing transverse momentum} \label{sec:reconstruction_jetmet}
After the tracks and ECAL clusters belonging to muons, electrons, and isolated photons have been identified and masked, charged and neutral hadrons are formed out of
HCAL clusters and any remaining tracks and ECAL clusters linked to them. Any HCAL cluster in the tracker acceptance without a linked track becomes a neutral hadron.
Outside the tracker acceptance, any HCAL clusters along with any linked ECAL clusters become hadrons\footnote{All HF HAD clusters become HF hadrons.}
(no distinction can be made between charged and neutral hadrons in this region). Any HCAL cluster with a link to a track gives rise to a charged hadron.
If there is excess calorimeter cluster energy in a charged hadron candidate relative to its track momentum,
the part of the excess that the ECAL can account for becomes a nonisolated photon, and any remaining excess becomes a neutral hadron.

As a consequence of parton showering and hadronization, a single radiated quark or gluon from a hard-scatter event does not reach the CMS detector
as a single particle, but rather a collimated ``jet'' of hadrons (mostly pions) and their decay products (such as photons from neutral pion decays\footnote{A neutral
pion in a jet is likely to be reconstructed by the PF algorithm as a nonisolated photon.}).
The goal of a jet reconstruction algorithm is to group together the final-state particles that arose in such a manner from a common parent particle, such
that properties of the parent may be faithfully inferred.

The \textit{anti-kT} algorithm~\cite{ref:1126-6708/2008/04/063} begins by defining a set of distances $d_{ij}$, $d_{iB}$ between any two objects (particles or pseudojets) $i,j$
and between any object $i$ and the beam, respectively. A step of the algorithm starts by identifying the smallest such distance. If it is a $d_{ij}$, the two objects
$i,j$ are merged into a single pseudojet, while if it is a $d_{iB}$, the object $i$ is classified as a jet, its particles are removed from consideration, and all the
distances are recalculated. The process is repeated until all objects are removed, i.e.\ assigned into jets. The distance measures $d_{ij}$, $d_{iB}$ are defined by
\begin{equation}
\begin{gathered}
\begin{aligned}
\Delta_{ij}^{2} &= (y_{i} - y_{j})^{2} + (\phi_{i} - \phi_{j})^{2}\\
d_{ij} &= \min(p_{\mathrm{T},i}^{-2}, p_{\mathrm{T},j}^{-2})\frac{\Delta_{ij}^{2}}{R^{2}}\\
d_{iB} &= p_{\mathrm{T},i}^{-2}
\end{aligned}
\end{gathered}
\label{eq:jet_distance}
\end{equation}
where $y$, $\phi$, and $p_{\mathrm{T}}$ are defined as in sec.~\ref{sec:LHCCMS_CMS_coordinates} and $R$ is a free parameter governing the radius of a jet
in $y$--$\phi$ space\footnote{When all particles involved have a sufficiently high energy, the distance $\Delta R$ between any two particles in the $\eta$--$\phi$ plane
is approximately equal to $\Delta_{ij}$, and so a region of fixed $\Delta_{iP}$ around some particle $P$ is essentially the same as a cone of fixed $\Delta R$.}.

The anti-kT algorithm is an example of a sequential recombination algorithm, which have the property of being infrared and collinear (IRC) safe~\cite{ref:epjc/s10052-010-1314-6}.
Broadly speaking, this means that radiated particles with either very low energy (infrared) or very close alignment to the parent particle (collinear) are
consistently assigned to the same jet as the parent, and do not affect the final number of reconstructed jets. Jets assembled by the anti-kT algorithm
also tend to have smooth, circular boundaries in the $\eta$--$\phi$ plane, facilitating the experimental calibration of jet energies.
For these reasons, as well as the relative simplicity of its definition, the anti-kT algorithm has become the jet algorithm of choice for LHC experiments.
A detailed review and comparison of different jet algorithms, including a discussion of the issues of IRC safety, is given in~\cite{ref:epjc/s10052-010-1314-6}.

A jet is subsequently defined in this thesis as an anti-kT jet with $R = 0.4$ (\textit{AK4 jets}). The energy and momentum of a jet are both defined to be the sum of those
of its constituent particles. Jets may be reconstructed from observed experimental data by using PF particles as the input (\textit{PF jets}). Alternatively, jets may be formed
from only charged-particle tracks (\textit{track jets}), and these are used to identify the leading \textit{primary vertex} (PV) in an event. A PV is the point at which a hard parton--parton
collision occurred, but under conditions of high pileup, there may be many PVs. The leading PV is chosen as that with the largest sum of $\pT^{2}$ of all associated track objects,
which include track jets as well as missing transverse track momentum.

Charged hadrons may be excluded from consideration in PF jet clustering if they are not matched to the leading PV. This procedure, known as \textit{charged hadron subtraction} (CHS)~\cite{ref:CMS-PAS-JME-14-001},
significantly reduces the variability of PF jet energy as a function of pileup. The energy of neutral hadrons from pileup is estimated as a function of
the jet area combined with the average pileup energy density $\rho$, and this contribution is then also subtracted from the jet energy. A final set of jet energy corrections
corrects for misreconstructed or lost particles in the jet.

The negative vector sum of the transverse momenta of all PF particles in an event is called the PF \vecMET. The vector sum of the transverse momenta of
all true particles in an event must be zero, but not all particles can become PF particles. The PF algorithm does not attempt to reconstruct particles that do
not interact with the CMS, such as neutrinos (or DM or gravitons, in BSM theories). The PF \vecMET\ gives an estimate of the sum of the transverse
momenta of all such ``invisible'' particles in an event.

If PF particles are replaced by PF jets, the PF \vecMET\ will equal the negative vector sum of all the PF object transverse momenta in the event, where in this case
a PF object is either a PF particle or jet. But if a jet has its energy altered by jet-level calibrations after PF reconstruction,
that equality no longer holds. In order to restore the net transverse momentum balance in events with jets, the PF \vecMET\ is corrected by subtracting the total vectorial transverse
momentum change induced in PF jets by jet energy corrections.

Uncertainties in the magnitude of the calibrated jet energy scale must be propagated in a correlated
fashion to the corrected PF \vecMET. Photon energy scale uncertainties also introduce a corresponding \vecMET\ uncertainty, which can be substantial in events with high-\pT\ photons.
For the jets and photons considered in this thesis, both sources of uncertainty are between 1 and 2\% of the corresponding object's energy~\cite{ref:1748-0221/12/02/P02014, ref:CMS-DP-2018-028, ref:1748-0221/10/08/P08010}.
The impact on final analysis results is estimated by varying these objects' energies by their level of uncertainty, while simultaneously varying the \vecMET\ by a counterbalancing amount.
